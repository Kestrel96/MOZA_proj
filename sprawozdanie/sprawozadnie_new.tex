\documentclass{article}
\usepackage[english,polish]{babel}
\selectlanguage{polish}

\title{\huge  \Huge \textbf{MOZA Projekt} \\ \textbf{Wzmacniacz Kaskodowy 4 (wariant B)}}
\date{\today}
\author{ \LARGE Jakub Półtorak}

\usepackage{amsmath}
\usepackage{amsfonts}
\usepackage{graphicx}
\usepackage[T1]{fontenc}

\begin{document}
\maketitle
\pagenumbering{gobble}
\newpage
\pagenumbering{arabic}
\tableofcontents



\begin{center}
	\title{ \huge \textbf{Etap 1}}
\end{center}


\section{Opis problemu}
\section{Sformułowanie matematyczne zadań optymalizacji}
\section{Metody rozwiązania numerycznego}
\subsection*{Wybór punktu startowego}

\pagebreak
\begin{center}
	\title{ \huge \textbf{Etap 2}}
\end{center}

\section{Funckje celu i ograniczeń, skalowanie}
\section{Wykorzystane algorytmy}
\subsection{Interior Point}
\subsubsection*{Jakość, zbieżność}
\subsection{Patternsearch}
\subsubsection*{Jakość, zbieżność}
\section{Porównanie punktu startowego, optymalnego i zbioru Pareto}
\section{Podsumowanie}
\begin{itemize}
	\item Czy zadania optymalizacji sformułwoano prawidłowo?
	      \begin{itemize}
		      \item tak
	      \end{itemize}
	\item Czy uzsykano widoczną poprawę właśności obiektu?
	      \begin{itemize}
		      \item tak
	      \end{itemize}

	\item Jaka jest złozoność obliczeniowa procesu optymalizacji?
	      \begin{itemize}
		      \item hahahah
	      \end{itemize}

\end{itemize}



\end{document}


